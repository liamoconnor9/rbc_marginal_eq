%---------------------------------------------------------------------------%
\lecture{Research Presentation}{lec_present_method}
%---------------------------------------------------------------------------%
\section{\enorcn{Methods}{研究方法}}
%---------------------------------------------------------------------------%
\begin{frame}[fragile]
    \frametitle{Rayleigh's Linear Stability Analysis}
    \begin{itemize}
        \item Assume variables can be decomposed like 
        \begin{align}
            \vec{u}(x, z, t) &= \vec{u'}(x, z, t) \label{EQ:reynolds_dc_u} = u'(x, z, t)\hat{x} + w'(x, z, t)\hat{z} \\
            T(x, z, t) &= \overline{T}(z, t) + T'(x, z, t) \label{EQ:reynolds_dc_T}\\
            p(x, z, t) &= \overline{p}(z, t) +  p'(x, z, t). \label{EQ:reynolds_dc_p}
        \end{align}
        \item The Eigenvalue Problem (EVP) is given by
        \begin{align}
            \nabla \cdot \vec{u'} &= 0 \label{EQ:linear1}\\
            \frac{\partial\vec{u'}}{\partial t} &= - \nabla p' + T'\hat{z} + \mathcal{R} \nabla^2 \vec{u'} \label{EQ:linear2}\\
            \frac{\partial T'}{\partial t} + \frac{\partial \overline{T}}{\partial z} w' &= \mathcal{P} \nabla^2 T' \label{EQ:linear3}
        \end{align}
    \end{itemize}
    
\end{frame}

\begin{frame}[fragile]
    \frametitle{Solutions to the Linear Problem}
        Solutions are given by
        \begin{align}
            w'(x, z, t) &= A\, \Re\left[W(z) \, e^{i(k_xx-st)}\right] \label{EQ:normal_modes1}\\ 
            u'(x, z, t) &= A\, \Re\left[U(z) \, e^{i(k_xx-st)}\right] \label{EQ:normal_modes2}\\ 
            T'(x, z, t) &= A\, \Re\left[\theta(z) \, e^{i(k_xx-st)}\right] \label{EQ:normal_modes3}\\ 
            p'(x, z, t) &= A\, \Re\left[P(z) \, e^{i(k_xx-st)}\right] \label{EQ:normal_modes4}
        \end{align}
        where $A$ is the (undetermined) mode amplitude, $s = \omega + i\sigma$ is the eigenvalue, \newline
        
        and the allowed wavenumbers $k_x \in \left\{\frac{n\pi}{2} \, \big| \, n \in \mathbb{N}\right\}$.
    
\end{frame}

\begin{frame}[fragile]
    \frametitle{The Quasilinear Model}
    \begin{itemize}
        \item The 1D quasilinear initial value problem (IVP)
        \begin{equation}
            \frac{\partial \overline{T}}{\partial t} + \frac{\partial}{\partial z} \langle w'T' \rangle_x = \mathcal{P} \frac{\partial^2 \overline{T}}{\partial z^2} \label{EQ:T0_IVP}
        \end{equation}
        \item Evolution due to \textbf{Advection}: $\frac{\partial}{\partial z} \langle w'T' \rangle_x $
        and \textbf{Diffusion}: $\mathcal{P} \frac{\partial^2 \overline{T}}{\partial z^2}$\newline

        \item Derived by manipulating the nonlinear and linear equations\newline
        
        \item Can be solved in conjunction with the EVP provided the amplitude $A$ is known\newline
        
        \hspace{0.5cm} $\mathbf{\longrightarrow}$ \textbf{Marginal Stability}

    \end{itemize}
    
    
\end{frame}

\begin{frame}[fragile]
    \frametitle{Marginal Stability Constraint}
    \begin{itemize}
        \item We assume the perturbations evolve on a shorter timescale than the background state\newline
        
        \item We impose \textbf{Marginal Stability} at each timestep
        \begin{equation}
            \max_{k_x} \{ \sigma \} = 0.
        \end{equation}\newline

        \item The growth rate $\sigma$ is obtained by solving the EVP for some $\overline{T}(z,t)$\newline
        
        \item Does not allow us to solve for $A$ directly $\longrightarrow$ discrete root-finding methods
    \end{itemize}
    
\end{frame}

\begin{frame}[fragile]
    \frametitle{Solving for the Perturbation Amplitude $A$}
    \begin{enumerate}
        \item Given some guess for the amplitude $A$ and a fixed timestep $\Delta t$\newline
        
        \item We evolve the initial (marginally stable) temperature profile $\overline{T}(z,t) \; \longrightarrow \; \overline{T}(z,t+\Delta t)$\newline
        
        \item Solve the EVP using $\overline{T}(z,t+\Delta t)$\newline
        
        \item This renders $\sigma(A)$\newline
        
        \item Use Newton's method with finite differences to solve $\sigma(A) = 0$
    \end{enumerate}
    
\end{frame}

\begin{frame}[fragile]
    \frametitle{The Amplitude Guess}
    \begin{itemize}
        \item Consider advection and diffusion separately on $t\;\to\; t+\Delta t$\newline
        
        \item Solve $\frac{\partial\overline{T}}{\partial t} + \langle w'T' \rangle_x=0$, yielding $\overline{T}_{adv} \; \longrightarrow \; \sigma_{\rm{adv}}$\newline
        
        \item Solve $\frac{\partial\overline{T}}{\partial t} = \mathcal{P}\frac{\partial^2 \overline{T}}{\partial z^2}$, yielding $\overline{T}_{adv} \; \longrightarrow \; \sigma_{\rm{diff}}$\newline
        
        \item For small timesteps, we assume diffusion and advection act independently, i.e.
        \begin{equation}
            0 = \Delta \sigma \approx A^2\sigma_{\rm{adv}} + \sigma_{\rm{diff}} \quad \longrightarrow \quad A^2 \approx -\frac{\sigma_{\rm{diff}}}{\sigma_{\rm{adv}}}
        \end{equation}
    
    \end{itemize}
    
\end{frame}
%---------------------------------------------------------------------------%
